% !TEX encoding = UTF-8
% !TEX TS-program = pdflatex
% !TEX root = ../Tesi.tex
% !TEX spellcheck = it-IT

%************************************************
\chapter{Metodi Java non deterministici}
\label{metodi-java-non-deterministici}
%************************************************
	Nel capitolo Takamaka \ref{takamaka-chapter}, così come in Ethereum \ref{ethereum-chapter}, abbiamo descritto le transazioni come delle operazioni sullo stato eseguite in modo deterministico. Ciò significa che ogni transazione modifica lo stato globale in modo calcolabile. Non può quindi esistere alcuna forma di entropia o casualità, nelle funzioni white-listed di Takamaka. Abbiamo concluso dicendo che nel caso necessitassimo di una qualsiasi forma di casualità, questa dovrà essere fornita al di fuori della blockchain tramite l'utilizzo di Oracoli. L'esecuzione di uno smart contract dovrà quindi portare al medesimo cambiamento di stato a tutti coloro che lo eseguiranno, dato il contesto della transazione e lo stato della blockchain al momento di tale esecuzione. 

	Siamo quindi alla ricerca dei metodi Java che, in determinate situazioni, portano ad una forma di non-determinismo. Tali funzioni dovranno essere escluse dai metodi white-listed di Takamaka così da vietarne l'utilizzo durante la scrittura dei contratti. L'analisi comincia con uno dei package più utilizzati: \lstinline|java.lang|.

	\section{java.lang.Object}
		\lstinline|java.lang.Object| è la super classe di tutte la classi Java perciò queste erediteranno tutti i metodi di Object. I metodi che implementa sono numerosi e in certi casi, alcuni di questi, possono causare non determinismo.

		\subsection{Metodo hashCode()}
		\label{object-hashcode}
			Questo metodo restituisce l'hash code dell'oggetto su cui viene applicato. Ogni volta che viene invocato su di un oggetto, durante l'esecuzione di un'applicazione Java, deve restituire sempre lo stesso valore intero, a condizione che non vengano modificate delle variabili all'interno dell'oggetto stesso. Questo numero intero non deve rimanere coerente da un'esecuzione all'altra. Due oggetti risultano uguali in base al metodo \lstinline|equals(Object)|, quindi chiamare il metodo \lstinline|hashCode()| su ciascuno dei due oggetti deve ritornare lo stesso valore. Quindi, in definitiva, il metodo \lstinline|hashCode()| restituisce interi distinti per oggetti distinti. Ma vediamo come l'utilizzo di questo metodo su blockchain può riportare problemi:
				%
			\begin{lstlisting}
	public int foo() {
		return new Object().hashCode();
	}
			\end{lstlisting}
			%
			Come abbiamo detto qui sopra, il metodo \lstinline|hashCode()|, ritornerà valori uguali a patto che l'oggetto in questione non subisca cambiamenti. Tale valore non deve rimanere costante per ogni esecuzione e tanto meno in virtual machine diverse. Possono essere ritornati dei valori del tipo:
			%
			\begin{lstlisting}[numbers=none,frame=none]
	1627800613
	2065530879
	697960108
			\end{lstlisting}
			%
			L'utilizzo del metodo visto qui sopra sul sistema blockchain, può portare ad ottenere risultati diversi che possono influire sullo stato globale.
			
		\subsection{Metodo toString()}
			Questo metodo ritorna una stringa che rappresenta testualmente l'oggetto, più nel dettaglio viene ritornata una stringa composta dal nome della classe di cui l'oggetto è istanza seguita dal carattere @ e la rappresentazione esadecimale dell'hash code di tale oggetto. In altre parole, viene ritornato:
			%
			\begin{lstlisting}[numbers=none,frame=none]
	getClass().getName() + '@' + Integer.toHexString(hashCode())
				\end{lstlisting}
			%
			Il problema può sorgere quando si tenta di implementare un metodo del tipo:
			%
			\begin{lstlisting}
	public String foo() {
		return new Object().toString();
	}
			\end{lstlisting}
			%
			Come visto nella sezione \ref{object-hashcode}, il valore della rappresentazione intera dell'hash ritornato dalla funzione \lstinline|hashCode()|, sarà diverso per ogni virtual machine, o meglio, ad ogni invocazione di \lstinline|foo()|. Questo implica che ogni qual volta che una transazione utilizzerà il metodo \lstinline|foo()| verrà ritornata una stringa diversa. Possono infatti essere ritornati dei valori del tipo:
			%
			\begin{lstlisting}[numbers=none,frame=none]
	java.lang.Object@77468bd9
	java.lang.Object@1f89ab83
	java.lang.Object@e73f9ac
			\end{lstlisting}
			%
			Se questo fosse possibile su blockchain, ogni miner che procede con l'esecuzione tale metodo ne ricava un risultato diverso che può potenzialmente influire sullo stato globale salvato nello storage.
	
		
	\section{java.util.HashSet}
		Questa classe implementa l'interfaccia \lstinline|Set| e supporta una tabella hash. Non fornisce alcuna garanzia sull'ordine di iterazione del set; in particolare, non garantisce che tale ordine rimanga costante nel tempo. Questa classe offre prestazioni a tempo costante per le operazioni di base quali \lstinline|add, remove, contains, size|, sul presupposto che la funzione hash disperda gli elementi in modo corretto tra bucket. L'iterazione su questo set richiede un tempo proporzionale alla somma delle dimensioni dell'istanza di \lstinline|HashSet| (numero degli elementi), sommata alla capacità di supporto dell'istanza \lstinline|HashMap| (numero di bucket). Pertanto sarà importante non impostare una capacità iniziale troppo alta se si dà molta importanza alle prestazioni. Da notare che questa implementazione non è sincronizzata. Se più thread accedono contemporaneamente ad un hash, e almeno uno di questi thread modifica il set, si deve procedere con una sincronizzazione esterna. Il set dovrebbe essere incapsulato all'interno di un metodo \lstinline|Collections.synchronizedSet| ed è consigliabile farlo già alla creazione dell'\lstinline|HashSet| in oggetto.
		
		Gli iteratori restituiti dal metodo \lstinline|iterator| di questa classe sono fail-fast: se il set viene modificato in qualsiasi momento dopo la creazione dell'iteratore, quest'ultimo genera l'eccezione \lstinline|ConcurrentModificationException|. Quindi di fronte a modifiche simultanee, l'iteratore fallisce rapidamente ed in modo pulito, piuttosto che rischiare di comportarsi in modo arbitrario e non deterministico in qualche tempo indeterminato nel futuro.
		
		\subsection{Metodo iterator()}
			Restituisce un iteratore sugli elementi di questo set. Gli elementi vengono restituiti in nessun ordine particolare. Come detto in precedenza, la classe \lstinline|HashSet| non garantisce che l'ordine con cui vengono ritornati gli elementi rimanga costante.
			%
			\begin{lstlisting}
	private HashSet h = new HashSet(); 
	...
	public Iterator getIterator() { 
		return h.iterator();
	}
			\end{lstlisting}
			%
			Supponiamo quindi di scrivere uno smart contract con all'interno il metodo \lstinline|getIterator()| riportato qui sopra, che ci da la possibilità di ritornare un iteratore dell'\lstinline|HashSet h|. Se l'ordine di iterazione non è garantito rimanere costante nel tempo, ma bensì possiamo trovarci difronte a due iteratori, ritornati dalla funzione di contratto \lstinline|getIterator()|, che iterano con un diverso ordine sugli elementi di \lstinline|h|, può portare al non determinismo su blockchain. Una cosa è certa, l'iteratore ritornato può comunque spaziare su tutti gi elementi di \lstinline|h| anche se in ordine diverso, ma partendo da due iteratori che propongono due ordinamenti diversi dello stesso hashset è possibile eseguire delle operazioni che ci portano a dei risultati diversi. Se tale risultato va ad influire sullo stato globale possiamo dedurre che non possiamo permettere l'utilizzo incontrollato di tale Classe.
					
	\section{java.util.HashMap}
		L'implementazione è basata su HashTable dell'interfaccia \lstinline|Map|. Questa implementazione fornisce tutte le operazioni opzionali delle map, oltre a permettere valori \lstinline|null| e chiavi \lstinline|null|. Questa classe non fornisce garanzie sull'ordine della mappa, in particolare, non garantisce che tale ordine rimanga costante nel tempo. Questa implementazione fornisce prestazioni costanti per operazioni di base (\lstinline|get| e \lstinline|put|), assumendo che la funzione hash disperda gli elementi in modo corretto tra i vari bucket.
			
		\subsection{Metodo values()}
			Ritorna una \lstinline|Collection| view di tutti i valori contenuti nella map. Le modifiche della mappa si riflettono sulla  \lstinline|Collection| e viceversa.
			%
			\begin{lstlisting}
	private static Map<String, String> h = new HashMap<>();
	...
	public Iterator getIterator() { 
		return h.values().iterator();
	}
			\end{lstlisting}
			%
			Se un contratto implementa il metodo \lstinline|getIterator()| questo ritornerà un iteratore della Collezione che rappresenta i dati contenuti nella nostra hashMap \lstinline|h|. Purtroppo, come riportato nei java docs, la classe \lstinline|HashMap|, così come \lstinline|HashSet|, non fornisce garanzie sull'ordine delle map, ma soprattutto, non è garantito che l'ordine rimanga costante nel tempo. Per questo motivo può portare al non determinismo su blockchain.
			
	\section{java.util.stream}
		Una \lstinline|Stream| è una sequenza di elementi che supporta operazioni di aggregazione sequenziali e parallele. Oltre a \lstinline|Stream| che è un flusso di riferimenti ad oggetti, esistono delle specializzazioni primitive per \lstinline|IntStream|, \lstinline|LongStream| e \lstinline|DoubleStream|. Per eseguire un calcolo, le operazioni di flusso sono composte in una \textit{stream pipeline}. Una stream pipeline è costituita da una \textit{sorgente} (che potrebbe essere un'array, una \lstinline|Collection|, una funzione di generazione, un canale I/O, ecc.), zero o più \textit{operazioni intermedie} (che trasformano lo stream in un altro stream, come per esempio \lstinline|filter(Predicate)|) ed infine, un'operazione \textit{terminale} (che produce un risultato o un side-effect, come per esempio \lstinline|count()| o \lstinline|forEach(Consumer)|). Gli stream sono "pigri" cioè i calcoli sui dati d'origine vengono eseguiti solamente quando viene avviata l'operazione terminale.
		\lstinline|Collection| e \lstinline|Stream|, pur presentando alcune somiglianze superficiali, hanno obiettivi diversi. Le \lstinline|Collection| hanno l'obiettivo principale di fornire l'accesso ed una gestione efficiente ai dati che contengono. Al contrario, gli \lstinline|Stream| non forniscono un mezzo per accedere direttamente o manipolare i loro elementi, sono invece interessati alla descrizione dichiarativa della sorgente e delle operazioni di calcolo che saranno eseguite in forma aggregata sulla sorgente.	Una pipeline di flusso può essere vista come una query sull'origine dello stream.
		
		Andiamo ora a vedere più da vicino come viene utilizzato uno stream e che influenza può avere sullo stile di programmazione Java. Come visto nell'articolo \cite{java-8-stream}, data la classe \lstinline|Album| cerchiamo di ricavare tutti gli album pubblicati prima dell'anno 2000 e stamparne i relativi autori. Vediamo la versione scritta in Java 7 e la possibile scrittura in Java 8 con l'utilizzo delle stream.
		%
		\begin{lstlisting}
	// Implementazione in Java 7
	for (Album album : albums)
		if (album.getYear() < 2000)
			System.out.println(album.getAuthor());
	
	// Implementazione in Java 8
	albums.stream()
		.filter(album -> album.getYear() < 2000)
		.map(Album::getAuthor)
		.forEach(System.out::println);
		\end{lstlisting}
		%
		Nel primo caso, prendiamo il primo elemento, questo viene processato per poi passare al successivo. Questo tipo di operazione si chiama iterazione esterna. Nel secondo caso ci facciamo restituire un oggetto stream dalla collezione \lstinline|albums|, sul quale applichiamo il metodo \lstinline|filter| che ci permette di considerare solamente gli album che soddisfano un certo criterio, tale criterio viene espresso tramite una lambda espressione. Sugli album che superano questa selezione, tramite \lstinline|map| e con un metodo reference, ci facciamo ritornare solamente l'autore e procediamo con lo stampare ciascuno di questi elementi. Riguardando le due implementazioni qui sopra appare evidente che lavoriamo dicendo allo stream cosa fare, non come farlo, di conseguenza ciò che abbiamo scritto è più simile alla descrizione del problema formulato in partenza, piuttosto che ad una metodologia per risolverlo. 
		
		Quella appena descritta è la modalità sequenziale ma da la possibilità di eseguire le stesse operazioni in parallelo. Dal punto di vista sintattico basta sostituire \lstinline|stream()| con il costrutto \lstinline|parallelStream()| in questo modo:
		%
		\begin{lstlisting}
	albums.parallelStream()
		.filter(album -> album.getYear() < 2000)
		.map(Album::getAuthor)
		.forEach(System.out::println);
		\end{lstlisting}
		%
		Questa nuova soluzione risponde comunque al problema formulato ma il risultato da lei ritornato non è deterministico in quanto l'ordinamento ritornato da questa procedura è potenzialmente diverso ad ogni esecuzione.
		
		\subsection{Metodo findAny()}
			Restituisce un oggetto \lstinline|Optional| che descrive qualche elemento dello stream, oppure un empty \lstinline|Optional| in caso di stream vuoto. \`E un'operazione terminale ed il comportamento di questa operazione è esplicitamente non deterministico, infatti è libera di selezionare un qualsiasi elemento nello stream. Questo per consentire prestazioni massime nelle operazioni parallele, con la conseguenza che più invocazioni successive della stessa procedura sulla medesima fonte potrebbero non restituire lo stesso risultato. Il metodo alternativo stabile è \lstinline|findFirst()|.
			%
			\begin{lstlisting}[breaklines=true]
	private String foo() {
		List<String> lst = Arrays.asList("Jhonny", "David", "Jack", "Duke", "Jill", "Dany", "Julia", "Divya");
		
		return lst.parallelStream()
			.filter(s -> s.startsWith("J"))
			.findAny().get();
	}
			\end{lstlisting}
			%
			L'esecuzione consecutiva della funzione sopraccitata \lstinline|foo()| restituisce risultati diversi. Ad ogni esecuzione di tale metodo ci possiamo aspettare risultati diversi. Riportiamo qui sotto i risultati rappresentati con il formato \lstinline|nome : n-returns|.
			%
			\begin{lstlisting}[numbers=none,frame=none]
	Jill: 7060, Jack: 2319, Julia: 91, Jhonny: 179, Jenish: 351
			\end{lstlisting}
			%
			Sostituendo \lstinline|findAny()| con il metodo \lstinline|findFirst()|, come riportato sulla descrizione JavaDoc, ci viene ritornato sempre il medesimo risultato, nel nostro caso \lstinline|Jhonny: 10000|.
			Abbiamo quindi la conferma che \lstinline|findAny()|, utilizzato soprattutto con \lstinline|parallelStream|, risulta un metodo non deterministico, quindi non utilizzabile all'interno di smart contract blockchain.
			
		\subsection{Metodo forEach(Consumer<? super T> action)}
		\label{subsection-forEach}
			Questo metodo esegue un'azione per ciascun elemento dello stream. Il comportamento di questa operazione è esplicitamente non deterministico.
			Per le pipeline delle parallelStream  questa operazione non garantisce il rispetto dell'ordine con cui si presentano inizialmente gli elementi dello stream. Il rispetto di tale ordinamento comprometterebbe i benefici introdotti dall'analisi parallela. Le analisi su ogni elemento possono pertanto essere eseguite in qualsiasi momento e da qualsiasi thread che si libera e sceglie l'elemento di cui prendersi cura. Se l'operazione eseguita accede in modifica allo stato condiviso, dovrà provvedere a fornire l'adeguata sincronizzazione. 
			
			Quella appena riportata è la descrizione che Oracle da al metodo \lstinline|forEach| sulla pagina che presenta le stream \cite{oracle-util-stream}. Tale documentazione esprime chiaramente il non determinismo del sopracitato metodo anche se applicato a stream sequenziali. I svariati tentativi di trovare degli esempi di funzionamento non deterministi sono purtroppo falliti, le varie funzioni che facevano uso delle stream su cui applicavano il metodo \lstinline|forEach|, nel nostro caso, hanno restituito sempre gli elementi nell'ordine in cui sono stati sottoposti, ma l'affermazione \textit{"Il comportamento di questa operazione è esplicitamente non deterministico"} risulta molto chiara e ciò non significa che non si comporterà mai in modo deterministico ma bensì che tale metodo non è obbligato ad essere deterministico. Per questo motivo, il non riuscire a trovare un esempio di funzionamento anomalo che provi il non determinismo, non può voler dire che tale metodo ritorni sempre i risultati aspettati, ma bensì che non possiamo contare su di esso, specialmente se tale metodo verrà utilizzato all'interno di applicazioni che devono obbligatoriamente essere deterministiche. Possiamo quindi concludere che tale metodo non può essere incluso nei metodi white listed di \textit{takamaka}.
			
		\subsection{Metodo forEachOrdered(Consumer<? super T> action)}
			Questo metodo esegue un'azione per ciascun elemento dello stream, in base all'ordine di incontro se lo stream ne ha definito uno. Questa operazione è terminale ed elabora gli elementi uno alla volta, nell'ordine di incontro. L'esecuzione dell'azione su di un elemento avviene prima di eseguire la stessa sull'elemento successivo, ma per ogni dato elemento l'azione può essere eseguita da qualsiasi thread che si libera e sceglie.
			
			Diversamente dal metodo \lstinline|forEach|, Sezione \ref{subsection-forEach}, questa implementazione ritorna gli elementi nello stesso ordine con cui si presentavano all'interno della sorgente di partenza. \lstinline|forEachOrdered| processa quindi gli elementi nello stesso ordine con cui si presentano all'interno nella sorgente iniziale indipendentemente dal fatto che lo stream sia di tipo sequenziale o parallelo. Ma vediamone gli effetti applicati ad un'operazione di ricerca stringhe parallela:
			%
			\begin{lstlisting}[breaklines=true]
	private void foo() {
		List<String> lst = Arrays.asList("Jhonny", "David", "Jack", "Duke", "Jill", "Dany", "Julia", "Jenish", "Divya");
			
		lst.parallelStream()
			.filter(s -> s.startsWith("J"))
			.forEach(e -> System.out.print(e + " "));
		
		lst.parallelStream()
			.filter(s -> s.startsWith("J"))
			.forEachOrdered(e -> System.out.print(e + " "));
	}
			\end{lstlisting}
			%
			L'esecuzione del metodo qui descritto \lstinline|foo()| può generare il seguente output:
			%
			\begin{lstlisting}[numbers=none,frame=none]
	Jill Jack Jenish Jhonny Julia 
	Jhonny Jack Jill Julia Jenish 
			\end{lstlisting}
			%
			Confrontando la lista \lstinline|lst| di partenza possiamo notare che gli elementi filtrati tramite \lstinline|filter(s -> s.startsWith("J"))|, nel primo caso vengono stampati in modo casuale, in base all'ordine con cui le varie thread hanno terminato l'analisi dei singoli elementi. Nel secondo caso notiamo invece che l'ordine con cui vengono ritornati i vari elementi risultato combacia con l'ordine in cui compaiono tali elementi nella lista sorgente. 
			
			Il metodo \lstinline|forEachOrdered|, diversamente da \lstinline|forEach|, garantisce che le lambda espressioni contenute al suo interno non vengano eseguite in contemporanea da più thread, oltre a dare la certezza che gli elementi vengano ritornati nello stesso ordine con cui sono stati sottoposti allo stream, tale metodo può quindi sostituire \lstinline|forEach| assicurando così un output deterministico. La conferma di ciò la possiamo vedere nel seguente esempio:
			%
			\begin{lstlisting}[breaklines=true]
	private void foo() {
		List<String> lst = Arrays.asList("Jhonny", "David", "Jack", "Duke", "Jill", "Dany", "Julia", "Jenish", "Divya");
	
		List<String> lst1 = new ArrayList<String>();
		lst2.parallelStream()
			.filter(s -> s.startsWith("J"))
			.forEach(e -> lst1.add(e));
			
		List<String> lst2 = new ArrayList<String>();
		lst2.parallelStream()
			.filter(s -> s.startsWith("J"))
			.forEachOrdered(e -> lst2.add(e));
	}
			\end{lstlisting}
			%
			Il metodo qui descritto ricerca con due procedure separate gli elementi all'interno della lista \lstinline|lst| che iniziano con il carattere \textit{J}, tali elementi verranno salvati tramite le due versioni di \lstinline|forEach| all'interno di due diverse liste, \lstinline|lst1| e \lstinline|lst2|. In particolare \lstinline|lst1| potrà anche non contenere tutti i risultati, questo dato dal fatto che \lstinline|forEach| viene eseguito in modo parallelo e, dato che ArrayList non è thread safe, se due thread tentano di salvare nello stesso momento due elementi nella medesima posizione, uno dei due risultati verrà perso. \lstinline|lst2| invece, conterrà sempre gli elementi \lstinline|[Jhonny, Jack, Jill, Julia, Jenish]| che appaiono nello stesso ordine con cui comparivano nella lista sorgente \lstinline|lst|.
			
			Possiamo sostituire \lstinline|forEach| con la la versione ordered sia con \lstinline|stream()| che con \lstinline|parallelStream()| ma, nel primo caso l'applicazione del \lstinline|forEachOrdered| implica uno sforzo computazionale di poco superiore al costo del \lstinline|forEach|. Nel secondo caso, se applicassimo \lstinline|forEachOrdered|, imponendo, oltre al rispetto dell'ordinamento nativo della sorgente, anche che tutte le operazioni sugli elementi vengano eseguite in modo seriale, comprometteremo i benefici introdotti dall'analisi parallela. 
			
			Concludendo, possiamo quindi affermare che il metodo \lstinline|forEachOrdered| si comporta sempre in modo deterministico dandoci la possibilità di eseguire delle operazioni parallele in modo deterministico ma compromettendone le altrimenti elevate prestazioni. 
		
		
		\subsection{Metodo peek(Consumer<? super T> action)}
			Ritorna uno stream costituito dagli elementi dello stream, eseguendo inoltre l'azione indicata su ogni elemento mano a mano che tali elementi vengono consumati dal flusso risultante. \lstinline|peek| risulta un'operazione intermedia. Se richiamata su di un \lstinline|parallelStream()| può essere eseguita in qualsiasi momento e da qualsiasi thread. Se l'operazione eseguita da \lstinline|peek| modifica lo stato condiviso si deve procedere con il fornire una adeguata sincronizzazione. 
			
			Come si evince dalla documentazione, \lstinline|peek| è un'operazione intermedia, vale a dire che non termina l'elaborazione del flusso, diversamente dalle operazioni esaminate nei capitoli precedenti, quali \lstinline|forEach| e \lstinline|forEachOrdered|. Questo metodo ci permette quindi di utilizzare un flusso senza terminare la pipeline delle operazioni agendo sui contenuti dello stream. Inoltre il metodo \lstinline|peek| è un'operazione che non interferisce, cioè garantisce che i dati dello stream sorgente non vengano modificati durante la sua esecuzione. L'obiettivo del metodo \lstinline|peek| è letteralmente quello di dare un'occhiata al contenuto dello stream. \`E stato infatti inserito all'interno delle API della classe \lstinline|Stream| per fornire un metodo per eseguire delle operazioni di debug. Questo metodo è perciò unico nel suo genere. Se eseguito in modo seriale, tramite \lstinline|stream()| analizza gli elementi uno ad uno e il suo comportamento sembra deterministico. Delle problematiche, come nel caso del metodo \lstinline|forEach|, sopraggiungono quando applichiamo \lstinline|peek| ad uno stream parallelo il cui ordine d'esecuzione cambia per ogni JVM. Il metodo \lstinline|peek| esegue delle lambda espressioni che possono, per esempio, salvare gli elementi dello stream all'interno di liste, più o meno thread safe. In definitiva questo metodo può essere molto utile sia per il debug che per eseguire analisi sui dati dello stream senza modificarne la struttura. Funziona in modo deterministico se applicato a stream seriali ma, se utilizzato su \lstinline|parallelStream()| il suo comportamento non è deterministico dato che l'ordine con cui analizzerà gli elementi può cambiare ad ogni esecuzione. Pertanto possiamo scegliere di accettarne l'utilizzo solamente se applicata ad una sorgente stream seriale.

						
		\subsection{Altri esempi di non determinismo}
		\label{esempi-non-det}
			Nel JavaDoc che descrive il package \lstinline|java.util.stream| viene scritto che ad eccezione delle operazioni identificate come esplicitamente non deterministiche, come i metodi \lstinline|findAny| e \lstinline|forEach|, se un flusso viene eseguito in sequenza o in parallelo non dovrebbe modificare il risultato del calcolo. Con un po' di fantasia però si può pensare ad una procedura non deterministica che stampi risultati diversi se eseguita in modo seriale piuttosto che parallelo.
			%
			\begin{lstlisting}[breaklines=true]
	private List<String> foo() {
		List<String> lst = Arrays.asList("Jhonny", "David", "Jack", "Duke", "Jill", "Dany", "Julia", "Divya");
	
		List<String> results = new ArrayList<>();
		lst.parallelStream()
			.filter(e -> e.startsWith("J"))
			.forEach(e -> results.add(e));
			
		return results;
	}
			\end{lstlisting}
			%
			La versione seriale, con l'utilizzo di \lstinline|stream()|, e quella parallela, qui sopra implementata, che utilizza \lstinline|parallelStream()| producono due risultati ben diversi. Quando utilizziamo uno stream parallelo, dobbiamo tenere sempre presente che introduciamo il non determinismo, come nella funzione \lstinline|foo()|, se cerchiamo in parallelo un elemento che abbia certe caratteristiche all'interno dello stream, e ce n'è più di uno, potremmo ottenere risposte diverse in esecuzioni diverse. Le righe precedenti, infatti, in maniera non prevedibile producono un'eccezione, dato che ArrayList non è thread-safe. L'arraylist \lstinline|results| potrà potenzialmente contenere i seguenti risultati:
			%
			\begin{lstlisting}[numbers=none,frame=none]
	[Jill, Jack, Julia, Jenish, Jhonny]
	[Jill, Jack, Julia, Jhonny, Jenish]
	[Jill, Jenish, Jack, Julia]
	...
			\end{lstlisting}
			%
			Una prima soluzione a questo problema può essere l'introduzione di una lista sincronizzata (thread-safe) che si prenda cura di garantire l'accesso in modo seriale alla struttura specificata. In questo modo se più thread tenteranno di accedere nello stesso momento alla lista, per aggiungere un nuovo elemento, solo una delle due thread riuscirà a salvare subito il nuovo dato, mentre l'altra resterà in attesa che la risorsa venga rilasciata. Nelle seguenti righe vediamo tale implementazione:
			%
			\begin{lstlisting}[breaklines=true]
	private List<String> foo() {
		List<String> lst = Arrays.asList("Jhonny", "David", "Jack", "Duke", "Jill", "Dany", "Julia", "Divya");
		
		List<String> results = Collections.synchronizedList(new ArrayList<>());
		lst.parallelStream()
			.filter(e -> e.startsWith("J"))
			.forEach(e -> results.add(e));
		
		return results;
	}
			\end{lstlisting}
			%
			La nuova lista \lstinline|result| ci assicura che tutti gli elementi trovati dalla procedura, in questo caso che tutti i nomi presenti nella lista \lstinline|lst| che iniziano con la lettere $J$, vengano riportati all'interno della lista \lstinline|results|. Questo modo di procedere però non assicura che l'ordinamento di \lstinline|results| resti fisso per tutte esecuzioni. L'accesso in scrittura a tale lista resta comunque legato all'ordine di terminazione delle varie thread che può variare ad ogni esecuzione. Questo ci porta a scartare anche questa soluzione. Vediamo alcuni potenziali output:
			%
			\begin{lstlisting}[numbers=none,frame=none]
	[Jill, Jack, Julia, Jenish, Jhonny]
	[Jill, Jenish, Julia, Jhonny, Jack]
	[Jill, Jhonny, Jack, Julia, Jenish]
	...
			\end{lstlisting}
			%
			Nelle seguenti righe vediamo un'altra variante che rende costante l'ordinamento di \lstinline|results|:
				%
			\begin{lstlisting}[breaklines=true]
	private List<String> foo() {
		List<String> lst = Arrays.asList("Jhonny", "David", "Jack", "Duke", "Jill", "Dany", "Julia", "Divya");
	
		List<String> results = lst2.parallelStream()
			.filter(e -> e.startsWith("J"))
			.collect(Collectors.toList());
	
	return results;
	}
			\end{lstlisting}
			%
			In un qualsiasi stream, parallelo o sequenziale, le operazioni contenute nella pipeline non devono modificare la sorgente dati, ad eccezione di alcune operazioni intermedie, inoltre le lambda che vengono passate, non devono dipendere dallo stato di qualche altro oggetto anche se esterno alla pipeline, devono quindi essere stateless. Se questi accorgimenti non vengono rispettati possiamo ottenere risultati non corretti o imbatterci in eccezioni a runtime.
			
		\subsection{Side-effects}
			Gli side-effects, o effetti collaterali, sono il risultato di quelle operazioni, all'interno degli stream, che modificano lo stato un qualche oggetto. Rappresenta tutte quelle modifiche dei valori dei campi di istanza o di classe eseguiti oltre al calcolo/restituzione di un valore. Ovviamente, un metodo che non ha un valore di ritorno deve avere una sorta di effetto collaterale che lo giustifichi. Nelle sezioni precedenti abbiamo appunto esaminato alcuni metodi della classe \lstinline|Stream| che producono side-effects come \lstinline|forEach|, \lstinline|forEachOrdered| e \lstinline|peek|. Questi metodi non avrebbero senso di esistere se non producessero side-effects, dato che, a parte \lstinline|peek|, non restituiscono alcun risultato. Si deve prestare molta attenzione quando si fa utilizzo di queste istruzioni, soprattutto in presenza di \lstinline|parallelStream()| è perciò consigliabile, dove del codice contiene inutilmente delle istruzioni che generano side-effects, di rimpiazzare queste istruzioni con altre operazioni di riduzione sicure.
			Possiamo vedere un esempio di tutto ciò nella sezione \ref{esempi-non-det}, dove si è preferito l'utilizzo dell'istruzione terminale \lstinline|.collect(Collection.toList())| che restituisce una lista di risultati, all'utilizzo di \lstinline|.forEach(s -> result.add(s))| che inseriva gli elementi risultato all'interno di una lista precedentemente dichiarata. In questo modo, evitando gli side-effects, la funzione si comporterà in modo deterministico, ritornando sempre il risultato atteso.
			
	\section{java.util.Random}
	\label{java.util.random}
		Un'istanza di questa classe può essere utilizzata per generare uno stream di numeri pseudocasuali a partire da un seed di 48-bit. Tale seed viene modificato ad ogni invocazione con l'utilizzo della formula \textit{LCF} (Linear Congruential Formula) ed utilizzato per il calcolo del successivo numero pseudocasuale. 
		\[ X_{n+1} = (aX_n + c) \quad \text{mod} \quad m\]
		dove il modulo $m > 0$, il moltiplicatore $0 < a < m$, l'incremento $0 \leq c < m$ e il valore di partenza $X_0$ rappresenta il seed e deve essere compreso tra $0 \leq X_0 < m$.
		Se due istanze di Random vengono create con lo stesso seed, queste produrranno la medesima sequenza di numeri pseudocasuali. Uno dei costruttori \lstinline|Random()| si prende cura di settare un valore di seed che, con molta probabilità, sarà distinto da qualsiasi altra chiamata. Il secondo costruttore \lstinline|Random(int seed)| crea anch'esso un nuovo generatore di numeri pseudo casuali ma settando il valore di seed, che influenzerà i valori generati, con il valore passato. 
		%
		\begin{lstlisting}
	private int rnd() {
		return new Random().nextInt(); 
	}
	
	private int rnd1() {
		return new Random(1993).nextInt(); 
	}  
		\end{lstlisting}
		%
		Il valore impostato dal costruttore vuoto \lstinline|Random()| non è prevedibile perché calcolato con il seguente calcolo:
		%
		\begin{lstlisting}[numbers=none,frame=none]
	public Random() {
		this(seedUniquifier() ^ System.nanoTime());
	}
	
	public Random(long seed) {
		if (getClass() == Random.class)
			this.seed = new AtomicLong(initialScramble(seed));
		else {
			// subclass might have overriden setSeed
			this.seed = new AtomicLong();
			setSeed(seed);
		}
	}
		\end{lstlisting}
		%
		Questo genererà un valore di seed diverso su ogni macchina che eseguirà il metodo \lstinline|rnd()| con la conseguente generazione di una sequenza di numeri pseudocasuali totalmente diversa. Tuttavia procedendo con la creazione di un'istanza della classe Random tramite il costruttore \lstinline|Random(int seed)|, fissato un \lstinline|int seed|, non accade tutto ciò, ma bensì ogni volta che si invocherà tale costruttore con il medesimo valore di \lstinline|seed|, verrà generata sempre la stessa sequenza di numeri pseudocasuali su qualsiasi JVM. Con l'unica restrizione che si utilizzi la stessa funzione LCF. 
		Non rappresenta un'implementazione di una funzione Random nel vero senso della parola, ma se si avesse bisogno di una specifica sequenza di numeri pseudocasuali, per una determinata funzione, la si può utilizzare. Un esempio di questa implementazione può essere \lstinline|rnd1()|, questo metodo ritorna semplicemente il valore restituito da \lstinline|nextInt()| che risulta essere sempre \lstinline|-1626801979|. Possiamo anche ritornare l'oggetto \lstinline|new Random(1993)|, anch'esso su qualsiasi altra JVM, applicando in modo consecutivo \lstinline|nextInt()|, si comporterebbe in modo deterministico restituendo sempre la sequenza:
		\[ \text{-1626801979, 1700409766, -290245901, -226742933, 1081295581, \dots}\]
		Le modalità con cui viene calcolato tale sequenza di interi sono descritte dal metodo invocato dal \lstinline|nextInt()| ovvero \lstinline|next(32)| all'interno della classe \lstinline|Random.java|.
		Concludiamo affermando che un'istanza di \lstinline|Random(int seed)| con \lstinline|seed| fissato, si comporta in modo deterministico su tutte le JVM, quindi si presta all'utilizzo per la programmazione della blockchain.
		
	\section{java.lang.System}
		La classe \lstinline|System| contiene una vasta gamma di campi e metodi utili e non può essere istanziato.
		
		\subsection{Metodo currentTimeMillis()}
			Questo metodo ritorna un \lstinline|long| che rappresenta il tempo corrente misurando i millisecondi trascorsi dalla mezzanotte del 01/01/1970 UTC. 
			Questo metodo è utilizzato da una vasta gamma di classi tra cui \lstinline|java.util.Date|, \lstinline|java.time.Clock| e \lstinline|java.util.Timer|. Supponiamo di implementare la seguente funzione: 
			%
		\begin{lstlisting}
	private long getCurrentMillis() {
		return System.currentTimeMillis();
	}
		\end{lstlisting}
			%
			Ogni JVM che eseguirà \lstinline|getCurrentMillis()| riceverà un risultato che varia in base all'istante in cui viene eseguito. I miner della network blockchain hanno tutti un'ordine potenzialmente diverso di esecuzione delle transazioni che vengono analizzate per ordine di arrivo e in base a delle logiche di guadagno, \lstinline|fee| maggiore $\Leftrightarrow$ priorità maggiore. Per questo motivo i miner possono analizzare una transazione in momenti diversi e nel caso una transazione richiamasse la funzione \lstinline|getCurrentMillis()|, ogni JVM può potenzialmente ricavarne valori diversi. Le soluzioni apportate, per esempio, in Ethereum sono rappresentate dall'utilizzo di oracoli esterni. Ma in questo caso abbiamo bisogno di un dato istantaneo e l'utilizzo di tali oracoli non costituirebbe comunque una soluzione dato che, come visto nella sezione \ref{ethereum-oracoli}, tale dato dovrebbe prima essere trasferito sulla catena rendendo i dati disponibili in memoria all'interno di uno smart contract. Ma questi passaggi non sono istantanei e necessitano di un certo tempo per essere eseguiti, perdendo la proprietà di freshness di tale dato. Un'alternativa adottata da Solidity è quella di utilizzare alcune proprietà del blocco corrente per ricavare un timestamp utile, come per esempio \lstinline|now|, alias di \lstinline|block.timestamp| che restituisce il tempo trascorso tra la creazione del blocco corrente e il 01/01/1970 00:00:00 UTC, restituita in secondi.
			Nel nostro caso non possiamo comunque accettare l'utilizzo del metodo \lstinline|System.currentTimeMillis()| e di tutti gli altri metodi che ne fanno uso.
		
		\subsection{Metodo System.nanoTime()}
			Questo metodo ritorna un valore temporale di esecuzione della JVM misurata in nanosecondi, può essere utilizzato solo per misurare il tempo che trascorso dato che non è in alcun modo correlato a nessuna sorgente temporale fornita dal sistema. 
			Questo metodo lo abbiamo incontrato in precedenza quando stavamo analizzando la classe \lstinline|Random| nella sezione \ref{java.util.random}, veniva infatti utilizzato per inizializzare il costruttore \lstinline|Random(long Seed)| nel caso di invocazione del costruttore vuoto \lstinline|Random()|. L'utilizzo del metodo \lstinline|nanoTime()|, è in quel caso, scusato dal fatto che è fonte di una buona casualità, utile infatti per trovare un numero abbastanza casuale. Ma vediamo come potrebbe comparire all'interno di uno smart contract:
			%
			\begin{lstlisting}
	private long getNanoTime() {
		return System.nanoTime();
	}
			\end{lstlisting}
			%
			Se un qualunque contratto richiamasse il metodo \lstinline|getNanoTime()| ogni JVM ne ricaverebbe un valore diverso con una buona fonte di casualità dato che è improbabile che due JVM siano attive dallo stesso lasso di tempo ed eseguano la stessa richiesta nel medesimo momento. \`E proprio questa fonte di casualità che ci porta a scartare tale metodo dall'utilizzo su blockchain, arrivando quindi alla medesima conclusione a cui siamo arrivati nella sezione precedente, per tale fonte di casualità, scartavamo la possibilità di poter utilizzare il costruttore vuoto della classe \lstinline|Random|. Possiamo quindi scartare l'utilizzo del metodo \lstinline|System.nanoTime()| e di tutte la classi che ne fanno uso.